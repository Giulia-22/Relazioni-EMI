
%----------------------------------------------------------------------------------------
%	PACKAGES AND OTHER DOCUMENT CONFIGURATIONS
%----------------------------------------------------------------------------------------



%% immagini
%\begin{figure}[h]
% \centering
% \includegraphics[width=\columnwidth]{diagrammiClassi/gerarchiaPersone}
% \caption{Diagramma delle classi - Gerarchia utenti.}
%\end{figure}

%----------------------------------------------------------------------------------------
%	ASSIGNMENT INFORMATION
%----------------------------------------------------------------------------------------
%% APPUNTI
% È
% \begin{figure}[h]
% \centering
% \includegraphics[width=16cm]{immagini/img1}
% \caption{Descr ~\cite{rif1}.}
%  \end{figure}


\documentclass{report}

\usepackage[utf8]{inputenc}
\usepackage[italian]{babel}
\usepackage{import}
\usepackage{todonotes}
\usepackage{color}
\usepackage{rotating}
\usepackage[hidelinks]{hyperref}
\usepackage{url}
\usepackage{pdfpages}
\usepackage{siunitx}
\usepackage{pdflscape}
\usepackage{subfig}
\usepackage[euler]{textgreek}

\usepackage{amsmath}
\usepackage{amsfonts}

\usepackage[signatures,swapnames,sans]{frontespizio}

\usepackage{geometry}
\geometry{portrait, margin=3cm}
\usepackage{siunitx}

\renewcommand*\figurename{Figura}

\newcommand{\sub}[1]{\textsubscript{#1}}
\newcommand{\super}[1]{\textsuperscript{#1}}

\newcommand{\Fig}[0]{Fig.}

\usepackage{titlesec}

\titleformat{\chapter}{\normalfont\huge}{}{20pt}{\huge\bfseries}

\begin{document}
\addtocounter{chapter}{-1}
	\begin{frontespizio}
		\Margini{3cm}{3cm}{3cm}{3cm}
		\Universita{Bergamo}
		\Logo[43.332mm]{unibg-mark}
		\Divisione{Scuola di Ingegneria}
		\Corso[Laurea Magistrale]{Ingegneria Informatica}
		\Titolo{Elettronica e Misure Industriali}
		\Sottotitolo{Relazione esperienze di laboratorio}
		\Punteggiatura{}
		\NRelatore{Prof.}{Prof.}
		\Relatore{Valerio Re}
		\NCorrelatore{Prof.}{Prof.}
		\Correlatore{Massimo Manghisoni}
		\Candidato[1058231]{Giulia Allievi}
		\Annoaccademico{2021--2022}
		\begin{Preambolo*}
			\usepackage[italian]{babel}
			\usepackage[T1]{fontenc}
			\usepackage[utf8]{inputenc}
			\usepackage{microtype}
			\usepackage{lmodern}
			\graphicspath{{img/}}
			
			\renewcommand{\frontinstitutionfont}{\fontsize{14}{17}\bfseries\scshape}
			\renewcommand{\fronttitlefont}{\fontsize{17}{21}\bfseries\scshape}
			\renewcommand{\frontfootfont}{\fontsize{12}{14}\bfseries\scshape}
		\end{Preambolo*}
	\end{frontespizio}



%----------------------------------------------------------------------------------------
%	PAGINA BIANCA
%----------------------------------------------------------------------------------------
\newpage
\null
\thispagestyle{empty}
\newpage

%----------------------------------------------------------------------------------------
%	INDICE
%----------------------------------------------------------------------------------------
\tableofcontents

%----------------------------------------------------------------------------------------
%	PAGINA BIANCA
%----------------------------------------------------------------------------------------
\newpage
\null
\newpage

%----------------------------------------------------------------------------------------
%	INTRO
%----------------------------------------------------------------------------------------
\chapter{Introduzione}

%----------------------------------------------------------------------------------------
%	CIRCUITO 1: EMITTER FOLLOWER	
%----------------------------------------------------------------------------------------
\chapter{Circuito 1: Emitter Follower}
\section{Introduzione} 
\section{Prima versione} % alimentazione duale
\subsection{Schema} 
\subsection{Analisi del circuito} 
\subsection{Componenti e misure} 
\section{Seconda versione} % alimentazione singola
\subsection{Schema} 
\subsection{Analisi del circuito} % versione base + versione partitore + versione condensatore
\subsection{Componenti e misure} 


%----------------------------------------------------------------------------------------
%	CIRCUITO 2: COMMON EMITTER AMPLIFIER
%----------------------------------------------------------------------------------------
\clearpage
\newpage
\chapter{Circuito 2: Common Emitter Amplifier}
\section{Introduzione} 
\section{Prima versione} % senza degenerazione emitter, solo trattazione teorica non realizzato in lab
\subsection{Schema} 
\subsection{Analisi del circuito} 
\section{Seconda versione} % con degenerazione emitter, alimentazione duale
\subsection{Schema} 
\subsection{Analisi del circuito} 
\subsection{Componenti e misure} 
\section{Terza versione} % con degenerazione emitter, alimentazione singola (in piccolo segnale aggiungi già il condensatore)
\subsection{Schema} 
\subsection{Analisi del circuito} 
\subsection{Componenti e misure} 


%----------------------------------------------------------------------------------------
%	CIRCUITI 3 E 4: AMPLIFICATORE OPERAZIONALE \mu A741
%----------------------------------------------------------------------------------------
\clearpage
\newpage
\chapter{Circuiti 3 e 4: Amplificatore operazionale \textmu A741}
\section{Introduzione} 
\section{Amplificatore invertente} 
\subsection{Schema} 
\subsection{Analisi del circuito} 
\subsection{Componenti e misure} 
\section{Integratore} 
\subsection{Schema} 
\subsection{Analisi del circuito} 
\subsection{Componenti e misure} 





%----------------------------------------------------------------------------------------

\end{document}
